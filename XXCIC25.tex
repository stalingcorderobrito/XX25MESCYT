\documentclass[11pt,a4paper]{article}
\usepackage[utf8]{inputenc}       % Codificación de caracteres
\usepackage[T1]{fontenc}
\usepackage[spanish]{babel}       % Idioma español
\usepackage{geometry}             % Márgenes ajustables
%\usepackage{hyperref}             % Hipervínculos
\usepackage{enumitem}             % Listas personalizadas
\usepackage{url}                  % Manejo de URLs
\usepackage{lmodern}              % Fuente vectorial
\usepackage{natbib}               % Gestión de citas bibliográficas
\usepackage[usenames,dvipsnames,svgnames,table]{xcolor}
\usepackage[colorlinks=true, linkcolor=black, urlcolor=blue, citecolor=green]{hyperref}
\definecolor{SCBCOLOR}{RGB}{50,110,30}
\definecolor{SCBGREEN}{RGB}{18,34,17}
\usepackage{svg}
\usepackage{tikz}
\usepackage{graphicx}
\usepackage{float}

\geometry{margin=1in}

\title{XX Congreso: Programa y Temáticas}
\author{}
\date{}

\begin{document}

\begin{titlepage}
    \centering
    % Ajusta el espacio superior según necesites
   % \vspace*{1cm}
    
    % Logo en la parte superior
    \includegraphics[width=\textwidth, height=\textwidth]{/Users/stalingcordero/Library/CloudStorage/GoogleDrive-scordero48@uasd.edu.do/Mi unidad/MESCyT/Viceministerio de CyT/Dirección de investigación/XXCIC25_visualcode/color.png} \\
    \vspace{2cm} % Espacio entre el logo y el texto
    
    % Bloque de texto centrado debajo del logo
    {\Huge \textbf{XX Congreso de Investigación Científica 2025} \par}
    \vspace{1cm}
    {\LARGE \textbf{Ministerio de Educación Superior, Ciencia y Tecnología} \par}
    \vspace{0.5cm}
    {\Large \textbf{Viceministerio de Ciencia y Tecnología} \par}
    
    \vfill  % Empuja el contenido para que quede centrado verticalmente respecto al espacio restante
\end{titlepage}


\maketitle
\tableofcontents
\newpage

\section{Magistrales}

\subsection{Estudio y Medida de la Equidad Educativa para la Mejora de la Calidad. Indicadores de Equidad en el Sistema Educativo de la República Dominicana}

%\noindent \textbf{Título:} \textit{Estudio y medida de la equidad educativa para la mejora de la calidad. Indicadores de equidad en el sistema educativo de la República Dominicana} \\

\subsubsection*{Dr. Fernando Martínez Abad}

\noindent Grupo de Investigación en InterAcción y eLearning (GRIAL) \\
Universidad de Salamanca, España


\subsubsection*{Resumen:} Tal y como destacan los Objetivos del Desarrollo Sostenible (ODS) y una creciente literatura científica la equidad educativa surge, junto a la eficacia escolar, como uno de los pilares clave para alcanzar una educación de calidad. Una educación equitativa y de calidad implica que el sistema educativo sea capaz de paliar o minimizar las brechas socioeconómicas y demográficas existentes, de modo que la escolarización contribuya al progreso de todas las personas y al adecuado funcionamiento del ascensor social. En este sentido, evaluaciones educativas a gran escala como las pruebas PISA nos ayudan a realizar un diagnóstico certero de los niveles de equidad y eficacia de los sistemas educativos nacionales y de su evolución a lo largo del tiempo. Es por eso que en los últimos 5 años está creciendo el interés científico sobre la evaluación de la equidad educativa, y es previsible que en los próximos años crezca aún más.Teniendo en cuenta la literatura actual en este ámbito, y la experiencia acumulada gracias a mi coordinación del Proyecto EVIDENCE, en esta conferencia se abordarán las siguientes cuestiones fundamentales:Conceptualización de la equidad educativa y sus dimensiones fundamentales: segregación, igualdad de resultados e igualdad de oportunidades.Técnicas estadísticas innovadoras y emergentes para la medida de la equidad educativa: indicadores para la medida de la segregación, para la medida de la igualdad de resultados y para la medida de la igualdad de oportunidades.Estudio de los indicadores de equidad en la educación secundaria de la República Dominicana en los años 2015, 2018 y 2022 (años en los que el país ha participado en las evaluaciones PISA): análisis en relación a los países de la región, de su evolución, y de los efectos del COVID-19.


\subsection{Avances de las Tecnologías de Fabricación Aditiva 3D en la Medicina Personalizada}

\subsubsection*{Prof. Dr. José L. Pedraz Muñoz}

\noindent Grupo de Investigación NanoBioCel, Laboratorio de Farmacia y Tecnología Farmacéutica, \\
Departamento de Farmacia y Alimentos, Facultad de Farmacia, \\
Universidad del País Vasco (UPV/EHU), Paseo de la Universidad 7, 01006 Vitoria-Gasteiz, España.

\subsubsection*{Resumen}

Durante las últimas décadas ha habido avances tecnológicos espectacu¬lares en el sector de la salud, como la modificación genética, la cirugía robótica, la nanotecnología, incluyéndose recientemente la impresión tridimensional (3D). Esta nueva área agrupa una serie de tecnologías aditivas que, aplicadas al sector sani¬tario, van a suponer un cambio en el paradigma de la atención a los pacientes. La impresión 3D, también conocida como fabricación aditiva, representa una revolución en el ámbito biomédico y farmacéutico desde principios de la década de 1980 como una técnica capaz de construir es¬tructuras tridimensionales y geometrías complejas capa por capa, basándose en diseños generados por ordenador. A partir de los años 2000, esta tecnología ha encontrado aplicaciones significativas en la industria farmacéutica, ingeniería de tejidos y medicina regenerativa, facilitando la creación de construcciones farma¬céuticas, biomédicas y biológicas tridimensionales con una precisión y compleji¬dad sin precedentes.  Se está en presencia de una tecnología de rápida evolución, con un gran potencial dirigido a la medicina personalizada y a la reducción de costes de producción. Las posibilidades de la impresión 3D en la producción farmacéutica y en la utili¬zación clínica de los medicamentos son diversas: pro¬ducir medicamentos a demanda, acelerar el desarrollo de los ensayos clínicos de nuevos medicamentos, facilitar la optimización de las formulaciones farmacéuti¬cas y desarrollar los tratamientos personalizados. Además, la fabricación de estructuras tridi-mensionales, mediante la deposición exacta de biomateriales y células vivas, po¬sibilita el control espacio-temporal sobre las interacciones celulares y la matriz extracelular. Esto dota a los constructos creados de la capacidad para imitar, no solo la estructura, sino también la funcionalidad de tejidos y órganos nativos, lo que abre nuevas vías para la recuperación funcional de órganos dañados o enfer¬mos. En resumen, las tecnologías aditivas 3D prometen transformar radicalmente el futuro de la medi¬cina. permitiendo la creación de tejidos vivos complejos, marcando el comienzo de una nueva era en el tratamiento de enfermedades y la reparación de tejidos. 

\subsection{COVID}

\begin{itemize}[leftmargin=*, label={--}]
    \item \textbf{Fernando Martínez Abat, PhD} --- Estadística y Educación.
    \item \textbf{Sten H. Vermund, MD, PhD} --- Vacunas, COVID, Secuelas y Modelos Epidemiológicos.
    \item \textbf{Giovannna Riggio, PhD} --- Bibliometría e Indicadores Bibliométricos.
    \item \textbf{José Luís Pedraz Muñoz, PhD} --- Farmacia e Ingeniería.
\end{itemize}


\section{Simposios}
\subsection{Tierras raras}

\subsection{Temáticas y Ponentes}
\begin{itemize}[leftmargin=*, label={--}]
    \item \textbf{Matemáticas} --- Juan Toribio Milané.
    \item \textbf{Biocombustibles} --- Yessica Castro.
    \item \textbf{Biología Marina} --- Yira A. Rodríguez.
    \item \textbf{Capacidades en Infraestructura y Recursos Humanos en Biología Molecular (y Biotecnología)} --- Edian Franklin Franco.
    \item \textbf{Metodología de Enseñanza y Aprendizaje en Carreras de la Salud en Latinoamérica: Enfoques y Desafíos} --- Msuricio Soto Suazo.
\end{itemize}

\subsection{Temáticas Específicas}
\begin{itemize}[leftmargin=*, label={--}]
    \item \textbf{Filogenia de la Vida:} Estado actual, hipótesis en competencia y complementarias.
    \item \textbf{Biología Molecular:} Capacidades en Recursos Humanos y en Infraestructura.
    \item \textbf{Ciencia Abierta:} Concepto, calidad, OER, OAI y ética.
    \item \textbf{Ciencia de la Sostenibilidad:} ¿Existe una ciencia de la sostenibilidad?
    \begin{itemize}[leftmargin=1.5em, label={--}]
        \item Concepto.
        \item Historia, presente y futuro.
        \item \textbf{Una Sola Salud:} Concepto, alcances, disciplinas integradas y su papel en la República Dominicana.
        \begin{itemize}[leftmargin=1.5em, label={$\ast$}]
            \item Interdisciplinariedad.
            \item Filogenia.
            \item Microbiología.
            \item Biología Molecular.
            \item Tratamiento personalizado.
            \item Estado actual de las investigaciones científicas.
            \item Técnicas diagnósticas.
            \item Enfermedades crónicas.
            \item Enfermedades transmisibles.
            \item Modelos epidémicos.
        \end{itemize}
    \end{itemize}
    \item \textbf{Investigación Científica y Desarrollo.}
    \item \textbf{Inteligencia Artificial:} Pasado, presente, futuro; ética, humanidad, educación e investigación.
    \item \textbf{Agricultura de Precisión:} Perspectivas en la República Dominicana.
    \item \textbf{Morfometría e Imagenología} --- (Willy Maurer, IEESL).
    \item \textbf{La Epistemología y el Desarrollo de la Ciencia.}
    \item \textbf{Presencia de la Comunidad Científica de la República Dominicana:} Participación en revistas Q1-Q4 y producción científica.
    \item \textbf{Revistas Científicas en la República Dominicana:} Estado actual, calidad y perspectivas.
    \item \textbf{Desarrollo de las Matemáticas en la República Dominicana, siglo XXI.}
    \item \textbf{Semiconductores en la República Dominicana:}
    \begin{itemize}[leftmargin=1.5em, label={--}]
        \item Capacidad en Recursos Humanos e Infraestructura.
        \item Perspectivas.
        \item Formación de Recursos Humanos.
    \end{itemize}
    \item \textbf{Educación, Ciencia, Tecnología e Innovación:}
    \begin{itemize}[leftmargin=1.5em, label={--}]
        \item Innovación de Base Científico-Tecnológica.
        \item Transferencia de Tecnología.
    \end{itemize}
    \item \textbf{Rompecabezas:} Lógica, Matemáticas, Ciencia y Educación.
    \item \textbf{Teorías Conspirativas:} Concepto, importancia, origen, intención, impacto, psicología y sociología.
    \item \textbf{Expediciones Científicas en el Siglo XXI.}
\end{itemize}

\section{Paneles}
\begin{itemize}[leftmargin=*, label={--}]
    \item \textbf{Factibilidad de producir vacunas en la República Dominicana:} Infraestructura, recursos humanos, inversión y beneficios.
    \item \textbf{Construyendo el futuro espacial:} Perspectivas y desafíos para la República Dominicana (con la participación del Ministerio de Defensa, INTEC, MIREX e ITLA).
\end{itemize}

\section{Conferencias Especiales}
% Sección reservada para próximas inclusiones o detalles específicos.

\section{Cursos}
\begin{itemize}[leftmargin=*, label={--}]
    \item \textbf{Herramientas aplicadas en las tareas de investigación:} (Abdul, Staling, Willy, Edian, Carlos).
    \item \textbf{Análisis filogenético, conceptualización y métodos:}
    \begin{itemize}[leftmargin=1.5em, label={--}]
        \item Métodos.
        \item \textbf{Software:} Uso de algoritmos y estadísticas.
        \begin{itemize}[leftmargin=1.5em, label={$\bullet$}]
            \item \url{https://www.megasoftware.net/} (MEGA).
            \item \url{http://www.iqtree.org/} (IQ-Tree).
            \item \url{https://ugene.net/} (UGene).
        \end{itemize}
        \item \textbf{Bancos de datos:} Ejemplo, \url{https://www.ncbi.nlm.nih.gov/}.
    \end{itemize}
    \item Se propone elaborar una base de datos de literatura científica sobre los tópicos propuestos, para generar interacción dinámica entre los interesados y fomentar debates y discusiones durante el XX CIC.
\end{itemize}

\section{Evento Especial: Gestión de Residuos}
\subsection{Yessica Castro Estevez: \\
``De la Investigación a la Implementación: Soluciones Innovadoras en la Gestión de Residuos''}
\textbf{Fecha:} marzo 2025 \\
\textbf{Duración:} 2 horas \\
\textbf{Ubicación:} Por confirmar

\subsection{Contexto}
La gestión sostenible de residuos es un desafío ambiental crucial. Con el incremento de desechos y sus efectos negativos en el medio ambiente, se requiere transformar la investigación en soluciones prácticas. Se destacará el caso del sargazo y otros residuos orgánicos, mostrando cómo la ciencia puede convertir estos materiales en productos de valor (biocombustibles, bioproductos, energía).

\subsection{Objetivos del Evento}
\begin{enumerate}[label=\arabic*.]
    \item \textbf{Fomentar el Diálogo Intersectorial:} Espacio de discusión entre investigadores, gobiernos, sector privado y sociedad civil para aplicar la investigación en soluciones concretas.
    \item \textbf{Impulsar la Implementación de Soluciones Sostenibles:} Superar barreras que impiden la conversión de avances científicos en proyectos viables.
    \item \textbf{Generar Propuestas de Acción Concretas:} Identificar, mediante colaboración intersectorial, las acciones necesarias para proyectos escalables.
    \item \textbf{Promover la Colaboración Público-Privada:} Facilitar la cooperación entre los sectores para impulsar la economía circular y la sostenibilidad.
\end{enumerate}

\subsection{Estructura del Evento}
\begin{enumerate}[label=\arabic*.]
    \item \textbf{Conferencia Inicial:} 
    \begin{itemize}
        \item Tema: ``Gestión Sostenible de Residuos: Transformando Basura en Productos de Valor''.
        \item Se expondrán casos y proyectos innovadores enfocados en la valorización de residuos orgánicos.
    \end{itemize}
    \item \textbf{Panel de Discusión:} 
    \begin{itemize}
        \item Tema: ``De la Investigación a la Realidad: Implementando Soluciones de Gestión de Residuos''.
        \item Debate sobre las oportunidades y desafíos para materializar la investigación en acciones concretas.
    \end{itemize}
    \item \textbf{Perfiles de Panelistas:}
    \begin{itemize}
        \item Investigador/a en Gestión de Residuos o Tecnología Ambiental.
        \item Representante del Sector Público (Gobernanza y Políticas).
        \item Representante del Sector Privado (Industria de Gestión de Residuos o Energía Renovable).
        \item Experto/a en Desarrollo de Proyectos Ambientales.
    \end{itemize}
    \item \textbf{Sesión Interactiva:} Preguntas y respuestas con la audiencia.
    \item \textbf{Cierre y Conclusiones:} Resumen de propuestas y pasos a seguir para la implementación de soluciones sostenibles.
\end{enumerate}

\subsection{Audiencia Objetivo}
\begin{itemize}[leftmargin=*, label={--}]
    \item Investigadores y académicos en gestión de residuos y sostenibilidad.
    \item Autoridades y representantes del sector público.
    \item Empresas y actores del sector privado relacionados con la gestión de residuos y energías renovables.
    \item ONGs y organizaciones internacionales.
    \item Emprendedores y comunidades locales.
\end{itemize}

\subsection{Impacto Esperado}
\begin{itemize}[leftmargin=*, label={--}]
    \item \textbf{Transición de la Investigación a la Acción:} Facilitar la implementación de soluciones innovadoras basadas en investigaciones sobre residuos.
    \item \textbf{Fomento de la Colaboración:} Promover alianzas entre sectores público, privado y académico.
    \item \textbf{Aceleración de Proyectos Sostenibles:} Definir acciones concretas para convertir la investigación en proyectos viables y escalables.
\end{itemize}

\newpage

\addcontentsline{toc}{chapter}{Bibliografía}
\bibliographystyle{apalike}% estilo de la bibliografía.
\bibliography{biblio} % yyyy.bib es el fichero donde está salvada la bibliografía.


\end{document}
